
\chapter{Технологическая часть}\label{tecnology}
%\addcontentsline{toc}{chapter}{3 Технологическая часть}

\section{Требования к ПО}\label{Requirements}

Требования к программному обеспечению:
\begin{enumerate}
    \item на вход подается количество заявок;
    \item результат: замеры времени пребывания заявки на этапах и в очередях. 
\end{enumerate}

\section{Выбор языка программирования}\label{Language}

Был выбран язык go, поскольку он удовлетворяет необходимым требованиям и удобен программирования параллельных вычислений. 
Средой разработки выбрана Visual Studio Code.

\section{Структуры данных}\label{StructsList}

На листинге \ref{list:matrixstruct} представлено описание структуры заявки.

\begin{lstinputlisting}
    [caption = {Структура матрицы},
    label = {list:matrixstruct},
    linerange={30-39},
    ]{../Lab5_v2/request.go}
\end{lstinputlisting}

\section{Реализация конвейера}\label{Listings}

На листинге \ref{list:poslav} представлена реализация конвейера.

\begin{lstinputlisting}
    [caption = {Реализация последовательного алгоритма нахождения среднего арифметического матрицы},
    label = {list:poslav},
    linerange={15-84},
    ]{../Lab5_v2/conveer.go}
\end{lstinputlisting}

\section{Тестирование}\label{TestResult}


\textbf{Модульные тесты}

\begin{table}[ht]
    \caption{Тесты}
    \centering
\begin{tabular}{ l | l | l | l | l |}
    ${N^{\underline{o}}}$ & Ввод & Шифр Цезаря (key=13)&  Шифр Атбаш &  Шифр XOR (key a) \\ \hline \hline
    1 & abc & nop & zyx & yz\{ \\  \hline 
    2 & zyx & mlk & abc & bap \\  \hline 
\end{tabular}
\label{tab:matrixMultiply}
\end{table}

Тесты пройдены.

%~\section{Оценка трудоемкости}\label{Difficalties}~---
~\section{Вывод технологической части}\label{TechResults}~

Были реализованы исследуемые алгоритмы, программа прошла тесты и удовлетворяет требованиям.

