
\chapter{Технологическая часть}\label{tecnology}
%\addcontentsline{toc}{chapter}{3 Технологическая часть}

\section{Требования к ПО}\label{Requirements}

Требования к программному обеспечению:
\begin{enumerate}
    \item на вход подается ключ;
    \item результат: значение, соответствующее ключу. 
\end{enumerate}

\section{Выбор языка программирования}\label{Language}

Был выбран язык go, поскольку он удовлетворяет требованиям лабораторной работы. Средой разработки выбрана Visual Studio Code.

\section{Структуры данных}\label{StructsList}

На листинге \ref{list:matrixstruct} представлено описание структуры словаря.

\begin{lstinputlisting}
    [caption = {Структура словаря},
    label = {list:matrixstruct},
    linerange={10-13},
    ]{../Lab7/type.go}
\end{lstinputlisting}

\section{Реализация алгоритмов}\label{Listings}

На листинге \ref{list:poslav} представлена реализация алгоритма полного перебора.

\begin{lstinputlisting}
    [caption = {Реализация последовательного алгоритма нахождения среднего арифметического матрицы},
    label = {list:poslav},
    linerange={5-16},
    ]{../Lab7/proccess.go}
\end{lstinputlisting}

На листинге \ref{list:choise} представлена реализация алгоритма бинарного поиска.

\begin{lstinputlisting}
    [caption = {Реализация алгоритма бинарного поиска},
    label = {list:choise},
    linerange={18-29},
    ]{../Lab7/proccess.go}
\end{lstinputlisting}

На листинге \ref{list:insert} представлена реализация алгоритма формирования сегментов.

\begin{lstinputlisting}
    [caption = {Реализация алгоритма формирования сегментов},
    label = {list:insert},
    linerange={31-41},
    ]{../Lab7/proccess.go}
\end{lstinputlisting}

На листинге \ref{list:insert} представлена реализация алгоритма частотного анализа.

\begin{lstinputlisting}
    [caption = {Реализация алгоритма частотного анализа},
    label = {list:insert},
    linerange={43-56},
    ]{../Lab7/proccess.go}
\end{lstinputlisting}

\section{Тестирование}\label{TestResult}


\textbf{Модульные тесты}

В таблице представлены тестовые данные \ref{tab:matrixMultiply}.


\begin{table}[ht]
    \caption{Тесты}
    \centering
\begin{tabular}{ l | l | l }
    ${N^{\underline{o}}}$ & Ввод & Вывод   \\ \hline \hline
    1 &  0   &    Нет элемента.    \\  \hline 
    2 &  23400005678   &    Kathleen Johnson    \\  \hline 
    3 &  23400005678   &    2    \\ \hline 
    4 &  23401515678  &    Donna Weaver    \\  \hline 
    5 &  23420735678  &    Kenneth Hunter    \\  \hline  
\end{tabular}
\label{tab:matrixMultiply}
\end{table}

Тесты пройдены.

%~\section{Оценка трудоемкости}\label{Difficalties}~---
~\section{Вывод технологической части}\label{TechResults}~

Были реализованы исследуемые алгоритмы, программа прошла тесты и удовлетворяет требованиям.

