
\chapter{Технологическая часть}\label{tecnology}
%\addcontentsline{toc}{chapter}{3 Технологическая часть}

\section{Требования к ПО}\label{Requirements}

Требования к программному обеспечению:
\begin{enumerate}
    \item на вход подается матрица смежностей;
    \item результат: список вершин - кратчайший путь в графе. 
\end{enumerate}

\section{Выбор языка программирования}\label{Language}

Был выбран язык go, поскольку он удовлетворяет требованиям задачи. Средой разработки выбрана Visual Studio Code.

\section{Структуры данных}\label{StructsList}

На листинге \ref{list:myravsstruct} представлено описание структуры муравья.

\begin{lstinputlisting}
    [caption = {Структура муравья},
    label = {list:myravsstruct},
    linerange={11-16},
    ]{../Lab6_v2/myravs.go}
\end{lstinputlisting}

\newpage 

На листинге \ref{list:colonystruct} представлено описание структуры колонии.


\begin{lstinputlisting}
    [caption = {Структура колонии},
    label = {list:colonystruct},
    linerange={18-26},
    ]{../Lab6_v2/myravs.go}
\end{lstinputlisting}

\section{Реализация алгоритмов}\label{Listings}

На листинге \ref{list:poslav} представлена реализация алгоритма полного перебора.

\begin{lstinputlisting}
    [caption = {Реализация алгоритма полного перебора},
    label = {list:poslav},
    linerange={68-84},
    ]{../Lab6_v2/allsearch.go}
\end{lstinputlisting}

На листингах \ref{list:choise1}, \ref{list:choise2}, \ref{list:choise3} представлена реализация 
муравьиного алгоритма.

\begin{lstinputlisting}
    [caption = {Реализация муравьиного алгоритма ч.1},
    label = {list:choise1},
    linerange={85-122},
    ]{../Lab6_v2/myravs.go}
\end{lstinputlisting}
\begin{lstinputlisting}
    [caption = {Реализация муравьиного алгоритма ч.2},
    label = {list:choise2},
    linerange={122-159},
    ]{../Lab6_v2/myravs.go}
\end{lstinputlisting}
\begin{lstinputlisting}
    [caption = {Реализация муравьиного алгоритма ч.3},
    label = {list:choise3},
    linerange={159-191},
    ]{../Lab6_v2/myravs.go}
\end{lstinputlisting}

\section{Тестирование}\label{TestResult}


\textbf{Модульные тесты}

В таблице \ref{tab:matrixMultiply} представленны тесты.

\begin{table}[ht]
    \caption{Тесты}
    \centering
\begin{tabular}{ l | l | l }
    ${N^{\underline{o}}}$ & Ввод & Вывод   \\ \hline \hline
1&
\begin{tabular}{ l | l | l | l | l }
    0 &3& 1& 6& 8\\
    3 &0& 4& 1& 0\\
    1 &4& 0& 5& 0\\
    6 &1& 5& 6& 1\\
    8 &0& 0& 1& 1 
\end{tabular}
&
15
    \\  \hline 
    2&
    \begin{tabular}{ l | l | l | l  }
        0 &10 &15 &20\\
        10 &0 &35 &25\\
        15 &35 &0 &30\\
        20 &25 &30 &0\\
    \end{tabular}
    &
    80
    \\  
\end{tabular}
\label{tab:matrixMultiply}
\end{table}


Тесты пройдены.

%~\section{Оценка трудоемкости}\label{Difficalties}~---
~\section{Вывод технологической части}\label{TechResults}~

Были реализованы исследуемые алгоритмы, программа прошла тесты и удовлетворяет требованиям.

