
\chapter{Технологическая часть}\label{tecnology}
%\addcontentsline{toc}{chapter}{3 Технологическая часть}

\section{Требования к ПО}\label{Requirements}

Требования к программному обеспечению:
\begin{enumerate}
    \item наличие меню для реализации выбора пользователя;
    \item на вход подается матрица;
    \item результат: число -среднее арифметическое матрицы. 
\end{enumerate}

\section{Выбор языка программирования}\label{Language}

Был выбран язык C++, поскольку в нем используются нативные потоки. Средой разработки выбрана Visual Studio Code.

\section{Структуры данных}\label{StructsList}

На листинге \ref{list:matrixstruct} представлено описание структуры матрицы.

\begin{lstinputlisting}
    [caption = {Структура матрицы},
    label = {list:matrixstruct},
    linerange={21-25},
    ]{../LAB4_v2/matrix.hpp}
\end{lstinputlisting}

\section{Реализация алгоритмов}\label{Listings}

На листинге \ref{list:poslav} представлена реализация последовательного алгоритма нахождения среднего арифметического матрицы.

\begin{lstinputlisting}
    [caption = {Реализация последовательного алгоритма нахождения среднего арифметического матрицы},
    label = {list:poslav},
    linerange={46-55},
    ]{../LAB4_v2/matrix.cpp}
\end{lstinputlisting}

На листинге \ref{list:choise} представлена реализация распараллеленого по строкам алгоритма нахождения среднего арифметического матрицы.

\begin{lstinputlisting}
    [caption = {Реализация распараллеленого по строкам алгоритма нахождения среднего арифметического матрицы},
    label = {list:choise},
    linerange={56-67},
    ]{../LAB4_v2/matrix.cpp}
\end{lstinputlisting}

На листинге \ref{list:insert} представлена реализация распараллеленого по столбцам алгоритма нахождения среднего арифметического матрицы.

\begin{lstinputlisting}
    [caption = {Реализация распараллеленого по столбцам алгоритма нахождения среднего арифметического матрицы},
    label = {list:insert},
    linerange={68-79},
    ]{../LAB4_v2/matrix.cpp}
\end{lstinputlisting}

\section{Тестирование}\label{TestResult}


\textbf{Модульные тесты}


\begin{table}[ht]
    \caption{Тесты}
    \centering
\begin{tabular}{ l | l | l }
    ${N^{\underline{o}}}$ & Ввод & Вывод   \\ \hline \hline
    1 &
    0 0 
    &
    Нет элементов.
    \\  \hline 

    3 &
    1 0
    &
    Нет элементов.
    \\  \hline 
    
    2 &
    2 2 2 2 2 2 
    &
    2
    \\ 


\end{tabular}
\label{tab:matrixMultiply}
\end{table}

Тесты пройдены.

%~\section{Оценка трудоемкости}\label{Difficalties}~---
~\section{Вывод технологической части}\label{TechResults}~

Были реализованы исследуемые алгоритмы, программа прошла тесты и удовлетворяет требованиям.

